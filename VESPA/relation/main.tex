\documentclass[11pt,a4 paper]{article}
\usepackage{subfiles}

\usepackage{amsmath, amsthm}
\usepackage[english]{babel}
\usepackage[T1]{fontenc}
\usepackage[utf8]{inputenc}
\usepackage[margin=2cm]{geometry}
\usepackage{graphicx}
\usepackage{subfig}
\usepackage{caption}
\usepackage{siunitx}
\captionsetup{tableposition=top,font=small,width=0.8\textwidth}
\usepackage{booktabs}
\usepackage[table]{xcolor}
\usepackage[arrowdel]{physics}
\usepackage{mathtools}
\usepackage{tablefootnote}
\usepackage{amssymb}
\usepackage{enumitem}
\usepackage{mathcomp}
\usepackage{multicol}
\setlist[description]{font={\scshape}} %style=unboxed,style=nextline
\usepackage{wrapfig}
\usepackage{float}
\usepackage{import}
\usepackage{floatflt}
\usepackage{url}
\usepackage{commath}
\usepackage{bm}
\usepackage[version=4]{mhchem}
\usepackage{nicefrac}
\usepackage{ifthen}
\usepackage{comment}
\usepackage[colorinlistoftodos,textsize=tiny]{todonotes}
\usepackage{hyperref}

\renewcommand*{\thefootnote}{\fnsymbol{footnote}}
\sisetup{exponent-product = \cdot}
\newcommand{\tc}{\,\mbox{tc}\,}
\newcommand{\Epsilon}{\mathcal{E}}
\newcommand{\half}{\frac{1}{2}}
\newcommand{\overbar}[1]{\mkern 1.5mu\overline{\mkern-1.5mu#1\mkern-1.5mu}\mkern 1.5mu}
\let\oldfrac\frac
\renewcommand{\frac}[3][d]{\ifthenelse{\equal{#1}{d}}{\oldfrac{#2}{#3}}{\nicefrac{#2}{#3}}}

\title{VESPA\\Vaso per Esperimenti Su Plasmi ed Altro}
\author{Andrea Grossutti, mat. 1237344\\Alessandro Lovo, mat. 1236048\\Leonardo Zampieri, mat. 1237351}
\date{\today}

\begin{document}

\maketitle

\section{Aims}
Study the \emph{Vespa} experimental apparatus, and in particular:
\begin{itemize}[noitemsep]
    \item Model the vacuum system behavior, finding the characteristic parameters;
    \item Obtain the current-voltage and the current-temperature characteristics curves of the filament;
    \item Draw the voltage-current characteristics curves of the gas discharge, enhancing their behavior as varying pressure;
    \item Find the Paschen curve, both in DC and RF condition;
    \item Measurement of plasma parameters through a Langmuir probe, both in stationary conditions and via ionic-sonic wave propagation.
\end{itemize}


\section{Vacuum system}
The vacuum inside the VESPA vessel (a cylindrical  vessel, with a length of $\sim80\si{\centi\metre}$ and a radius of $\sim20\si{\centi\metre}$: $V\sim0.1\si{\metre^3}$) is obtained and kept thanks to a rotary pump and a turbomolecular pump. The vessel is not perfectly isolated and some small leaks affect the vacuum keeping. To study this phenomena, the vessel has been taken to a low pressure ($\sim 6 \cdot 10^{-5} \si{\milli\bar}$) and all the valves around have been closed. Isolating the chamber from the pumping system one can measure (thanks to a ionization pressure gauge) the pressure in the vessel as function of time. Effects as leaks and degasing contribute to an inflow in the chamber $F_0(p)$ that in principle could depend on the pressure. Assuming instead $F_0$ constant, its value can be estimated through a linear fit on the data: $P = a + b \cdot t$, $F_0 = V \cdot b$.

Considering the reaction time, the slowness of the ionization gauge in stabilizing and the pressure oscillations, the errors are estimated as $5\%$ on the pressure and a $0.5\si{\second}$ error on the time.

\begin{figure}[H]
  \centering
  \resizebox{0.5\textwidth}{!}{\import{img/vacuum/}{increase_all_fit.tex}} \hspace{-0.05\textwidth}
  \resizebox{0.5\textwidth}{!}{\import{img/vacuum/}{increase_all_residuals.tex}}
  \caption{Presure increasing in the chamber}
  \label{fig:increase_all}
\end{figure}

As can be seen from fig \ref{fig:increase_all} there is an evident trend in the residuals, proving that $F_0$ cannot be assumed constant throughout all the explored range of pressures. A simple way to correct this is to consider a low pressure regime and a high pressure one: the limit should be put where the trend in the residuals inverts, i.e. around 200\si{\nano\pascal}.

\begin{figure}[H]
  \centering
  \resizebox{0.5\textwidth}{!}{\import{img/vacuum/}{increase_lowP_fit.tex}} \hspace{-0.05\textwidth}
  \resizebox{0.5\textwidth}{!}{\import{img/vacuum/}{increase_lowP_residuals.tex}}
  \caption{Low pressure regime}
  \label{fig:increase_lowP}
\end{figure}

\begin{figure}[H]
  \centering
  \resizebox{0.5\textwidth}{!}{\import{img/vacuum/}{increase_highP_fit.tex}} \hspace{-0.05\textwidth}
  \resizebox{0.5\textwidth}{!}{\import{img/vacuum/}{increase_highP_residuals.tex}}
  \caption{High pressure regime}
  \label{fig:increase_highP}
\end{figure}

Splitting the high pressure area and the low pressure area, performing two different fits and assuming a 5\% error on the volume, the inflow can be estimated as:

\begin{gather*}
  F_0^\text{low} = (6.4 \pm 0.4) \cdot 10^{-6} \si{\pascal}\, \si{\metre^3}/\si{\second}, \quad
  F_0^\text{high} = (4.5 \pm 0.4) \cdot 10^{-6} \si{\pascal}\, \si{\metre^3}/\si{\second}
\end{gather*}

Subsequently, the valve has been opened connecting the chamber to the pumping system. An exponential decay of the pressure is expected: $P(t) = (P_i - P_0)\exp(-t/\tau) + P_0$, where $P_i$ is the starting pressure and $P_0$ the asymptotic pressure.

\begin{figure}[H]
  \centering
  \resizebox{0.5\textwidth}{!}{\import{img/vacuum/}{decrease_all_fit.tex}} \hspace{-0.05\textwidth}
  \resizebox{0.5\textwidth}{!}{\import{img/vacuum/}{decrease_all_residuals.tex}}
  \caption{Presure decreasing in the chamber}
  \label{fig:decrease_all}
\end{figure}

As can be seen from fig \ref{fig:decrease_all}, similarly as what seen before, the result are not acceptable and as before two regimes can be distinguished. For coherence the same limit as before has been used.

\begin{figure}[H]
  \centering
  \resizebox{0.5\textwidth}{!}{\import{img/vacuum/}{decrease_lowP_fit.tex}} \hspace{-0.05\textwidth}
  \resizebox{0.5\textwidth}{!}{\import{img/vacuum/}{decrease_lowP_residuals.tex}}
  \caption{Low pressure regime}
  \label{fig:decrease_lowP}
\end{figure}

\begin{figure}[H]
  \centering
  \resizebox{0.5\textwidth}{!}{\import{img/vacuum/}{decrease_highP_fit.tex}} \hspace{-0.05\textwidth}
  \resizebox{0.5\textwidth}{!}{\import{img/vacuum/}{decrease_highP_residuals.tex}}
  \caption{High pressure regime}
  \label{fig:decrease_highP}
\end{figure}

Performing two different exponential fits there are still trends in the residuals, but now the result is acceptable. From the values of $\tau$ and $P_0$ the effective pumping speed $S_e$, the inflow $F_0$ and, given the nominal value of the pumping speed $S = 33 \si{\litre/\second}$, the conductance of the chamber-pump connection $C$ can be estimated.

\begin{table}[H]
  \centering
  \begin{tabular}{cccccc}
    \toprule
    regime & $S_e = V/\tau$ [\si{\litre/\second}] & $F_0 = P_0 \cdot S_e$ [\si{\pascal}\, \si{\metre^3}/\si{\second}] & $C = 1/(1/S_e - 1/S)$ [\si{\litre/\second}] \\
    \midrule
    low pressure & $0.51 \pm 0.04$ & $(2.2 \pm 0.2)\cdot 10^{-6}$ & $0.52 \pm 0.05$ \\
    high pressure & $4.0 \pm 0.6$ & $(2 \pm 1)\cdot 10^{-5}$ & $4.6 \pm 0.7$ \\
    \bottomrule
  \end{tabular}
  \caption{Vacuum parameters}
  \label{tab:vacuum}
\end{table}

In the low pressure regime the two estimates of $F_0$ are not compatible but still comparable, on the other hand at high pressure the two estimates differ by an order of magnitude. By comparing the nominal value of the pumping speed with $S_e$ one can deduce that most of the pump potential is wasted by a very low conductance connection.

\section{Voltage-Current characteristic of the filament}
The filament inside the vessel is a tungsten filament with diameter $2r\sim0.25\si{{\milli\metre}}$ and length $L\sim10\si{\centi\metre}$. Combining Ohm law and emissivity rules, a theoretical characteristic curve can be obtained:
\[
    V = \frac{A^{\frac[f]{10}{7}}L}{\pi^{\frac[f]{13}{7}}r^{\frac[f]{23}{7}}(2\epsilon\alpha)^{\frac[f]{3}{7}}} \cdot I^{\frac[f]{13}{7}}
\]
where $\epsilon$ is the effective emissivity, $\alpha$ the StefanBoltzmann constant and $A$ a the resistivity proportional constant, such that the resistivity $\rho$ can be expressed as function of the temperature $T$ as
\[
    \rho(T) = AT^{\frac[f]{6}{5}}
\]

Pumping the vessel to a low pressure ($\sim 2 $), the voltage-current characteristic curve of the filament has been measured, producing the following data:

\begin{figure}[H]
  \centering
  \resizebox{0.8\textwidth}{!}{\import{img/voltage-current-filament/}{I137V_1.tex}}
  \caption{Voltage-Current characteristic for a filament; errors has been chosen as $0.3A^{13/7}$ and $0.1V$, due to the low sensibility of the measure system.}
  \label{fig:I137V_1}
\end{figure}

Fitting the data with a $V\propto I^{\frac[f]{13}{7}}$, the following parameters are found:
\begin{align}
    V &= m I ^{\frac[f]{13}{7}}\\
    m &= ( 0.391\pm0.002) \si{V\cdot A^{-\frac[f]{13}{7}}}
\end{align}
which lead to a value of
\[
    \epsilon \sim 0.2
\]
The $\chi^2$ confirm the meaningfulness of the fit; moreover, the effective emissivity ha a value similar to the typical ones (0.3).

Finally, the estimated filament temperature as a function of the driven current can be found:
\[
    T = \underbrace{\frac{A^{\frac[f]{5}{14}}}{\pi^{\frac[f]{5}{7}}r^{\frac[f]{15}{14}}(2\epsilon\alpha)^{\frac[f]{5}{14}}}}_k \cdot I ^{\frac[f]{5}{7}} \qquad \text{with } k\sim811\si{K\cdot A^{-\frac[f]{5}{7}}}
\]

\begin{figure}[H]
  \centering
  \resizebox{0.8\textwidth}{!}{\import{img/voltage-current-filament/}{temp_1.tex}}
  \caption{Projection of the filament temperature as function of the current}
  \label{fig:temp_1}
\end{figure}

\section{V-I characteristics of the discharge \& DC Paschen curve}
By polarizing the filament with respect to the grid (grounded) a discharge in the plasma can be achieved.
By measuring the polarization voltage on the power supply (0.1 \si{\volt} error) and the plasma current via the voltage fall on a resistor $R_{shunt} = (1 \pm 0.03) \si{\ohm}$, the discharge V-I curve can be studied. In particular the breakdown voltage is clearly visible.

\begin{figure}[H]
  \centering
  \resizebox{0.8\textwidth}{!}{\import{img/Paschen-DC/}{constP.tex}}
  \caption{V-I characteristics of the discharge at constant pressure $P$, varying the filament current $I_f$}
  \label{fig:constP}
\end{figure}

As can be seen from fig \ref{fig:constP} changing $I_f$ doesn't change the breakdown voltage, but changes the the characteristics of the plasma once it is ignited. This effect is due to the thermoionic emission of the filament and is explained by Richardson law:

\begin{equation*}
  I = \Sigma A T^2 \exp(-\frac{e\Phi}{k_BT})
\end{equation*}

where $A = 7\cdot 10^5 \si{\ampere/\metre^2\kelvin^2}$, $\Phi =  4.55 \si{\volt}$ and $\Sigma = 2\pi r L \sim 79$ \si{\milli\metre^2} is the surface of the filament.

By using the previous found relation between the temperature and the filament current $T = k\cdot I_f^{\frac[f]{5}{7}}$ the expected plasma current can be computed as:
\begin{gather*}
  I = JI_f^{\frac[f]{10}{7}}\exp(-QI_f^{-\frac[f]{5}{7}}) \qquad
  J \sim 3.6 \cdot 10^7 \text{A}^{-\frac[f]{3}{7}} \quad
  Q \sim 65 \text{A}^{-\frac[f]{5}{7}}
\end{gather*}

A first comparison between the predicted plasma current and the experimental one is to use for the latter the value at 60 \si{\volt} of polarization voltage: value at which the plasma has been ignited but the electric field is not very strong yet.

\begin{table}[H]
  \centering
  \begin{tabular}{cccccc}
    \toprule
    $I_f$ [\si{\ampere}] & $I^{\text{exp}}$ [\si{\ampere}] & $I^{\text{Rich}}$ [\si{\ampere}]  & $\frac[f]{I^{\text{Rich}}}{I^{\text{exp}}}$\\
    \midrule
    6.1 & 0.2 & 8 & 34 \\
    6.5 & 0.8 & 20 & 25 \\
    6.7 & 1.6 & 30 & 19 \\
    \bottomrule
  \end{tabular}
  \caption{Experimental plasma current at $V = 60$ \si{\volt} and prediction from Richardson law (approximate values)}
  \label{tab:Richardson}
\end{table}

As can be seen from tab \ref{tab:Richardson} the prediction and the data differ by a factor $\sim 25$, that is almost the same for every value of the filament current, suggesting that the main error in the estimate concerns coefficient $J$; or equivalently that the current emitted by the filament is not the plasma current, and this is due to electron absorption in the plasma. In this case the error in the prediction of the current could be an estimate of percentage of absorbed electrons ($\sim 96\%$). \todo[inline]{Hmm: it seems very high: comment}
On the other hand, Richardson model fails to predict other observed effects:
\begin{itemize}
  \item Once the plasma is ignited the plasma current increases with the polarization voltage, expecially at higher $I_f$. This is probably due to the strong electric field increasing the multiplication of electrons in the plasma.
  \item As the polarization potential increases, the voltage across the filament $V_f$ has also been observed to increase, meaning that the resistivity of the filament becomes greater. This means either that the resistivity does not depend only on the temperature fo the filament, or (more probably), that the temperature does not depend only on $I_f$.
\end{itemize}


By studying instead the V-I characteristics varying the Argon pressure one obtains the results in fig \ref{fig:constIf}. If one ignores the weird behavior of the $P = 8$ \si{\micro\bar} curve at high $V$, it is visible that the main influence of the pressure on the curve is the value of the breakdown voltage, effect that can be better understood building the experimental Paschen curve (fig \ref{fig:DCP}).

\begin{figure}[H]
  \centering
  \resizebox{0.8\textwidth}{!}{\import{img/Paschen-DC/}{constIf.tex}}
  \caption{V-I characteristics of the discharge at constant filament current $I_f$, varying the pressure $P$}
  \label{fig:constIf}
\end{figure}

\begin{figure}[H]
  \centering
  \resizebox{0.8\textwidth}{!}{\import{img/Paschen-DC/}{Paschen.tex}}
  \caption{Experimental Paschen curve: breakdown voltage vs $P$}
  \label{fig:DCP}
\end{figure}

As can be seen from fig \ref{fig:DCP} the curve is quite noisy so a fit with the theoretical formula (\ref{eq:paschen}) of the Paschen curve would fail.

\begin{gather} \label{eq:paschen}
  V_b = \frac{Bpd}{\log(Apd) - C} \qquad (pd)_{min} = \frac{e^{1+C}}{A}
\end{gather}

where $d$ is the distance between the electrodes (in this experiment the filament and the grid) and for Argon $A = 8.63 \si{\metre^{-1}\pascal^{-1}}$, $B = 132.0 \si{\volt\metre^{-1}\pascal^{-1}}$ and $C = 1.004$.

From fig \ref{fig:DCP} it is clear that Paschen's minimum is between 3 and 4 \si{\micro\bar}, but this would imply $d \sim 2.4$ \si{\metre}, that is rather unphysical considering that the whole chamber is only 0.8 \si{\metre} long. On the other hand using a more reasonable $d \sim 0.4$ \si{\metre} one obtains $p_{min} \sim 20$ \si{\micro\bar}.

It can then be deduced that this system does not follow the theoretical Paschen's curve with Argon values; but since no fit can be performed, nothing more can be said.

\todo[inline]{Other comments?}

\section{Paschen curve in radiofrequency condition}
Inside VESPA a magnetic antenna is provided, which allow to induce a plasma through radiofrequency waves that, coupling with gas electrons, excite them and generate the discharge.

The antenna is powered through a RF generator and a RLC circuit; firstly, the response of the circuit is studied. Keeping fixed the generator power and varying the frequency, through an oscilloscope the peak-to-peak amplitude of the wave downstream the circuit is measured. The result is a typical resonance courve (fig. \ref{fig:RF:response}).

\begin{figure}[H]
  \centering
  \resizebox{0.8\textwidth}{!}{\import{img/paschen-RF/}{circuit_response.tex}}
  \caption{Peak-to-peak amplitude downstream the circuit, varying the wave frequency.  From the oscillation in the measurements and the instrumentation sensibility, the errors has been estimated at $5\tcperthousand$ in the frequencies and $1\%$ in voltages.}
  \label{fig:RF:response}
\end{figure}

From the fig. \ref{fig:RF:response} the maximum can be detected: it is around $4\si{\mega\hertz}$. Therefore the operation area is identified in the range $3\textup{--}4\si{\mega\hertz}$, when the amplitude is high and increasing with the frequency.

Varying the pressure in the vessel, the breakdown voltage trend can be studied. For each pressure, the frequency has been increased until the discharge took place (it's a clearly visible phenomena, so the discrimination has been done visually), and the peak-to-peak voltage in the breakdown point has been measured. The result is the Paschen curve in fig. \ref{fig:RF:Paschen}.

\begin{figure}[H]
  \centering
  \resizebox{0.8\textwidth}{!}{\import{img/paschen-RF/}{paschen.tex}}
  \caption{Radiofrequency Paschen curve. From the oscillation in the measurements and the instrumentation sensibility, the errors has been estimated at $5\%$ in the pressures and $1\%$ in voltages.}
  \label{fig:RF:Paschen}
\end{figure}

As can be seen from the plot, the minimum of the Paschen curve in radiofrequency condition is very large: the optimal area can be identified in the range: $$3\textup{--}7 \cdot 10^{-5}\si{\bar}$$
















\end{document}
