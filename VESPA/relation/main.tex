\documentclass[11pt,a4 paper]{article}
\usepackage{subfiles}

\usepackage{amsmath, amsthm}
\usepackage[english]{babel}
\usepackage[T1]{fontenc}
\usepackage[utf8]{inputenc}
\usepackage[margin=2cm]{geometry}
\usepackage{graphicx}
\usepackage{subfig}
\usepackage{caption}
\usepackage{siunitx}
\captionsetup{tableposition=top,font=small,width=0.8\textwidth}
\usepackage{booktabs}
\usepackage[table]{xcolor}
\usepackage[arrowdel]{physics}
\usepackage{mathtools}
\usepackage{tablefootnote}
\usepackage{amssymb}
\usepackage{enumitem}
\usepackage{multicol}
\setlist[description]{font={\scshape}} %style=unboxed,style=nextline
\usepackage{wrapfig}
\usepackage{float}
\usepackage{import}
\usepackage{floatflt}
\usepackage{url}
\usepackage{commath}
\usepackage{bm}
\usepackage[version=4]{mhchem}
\usepackage{nicefrac}
\usepackage{ifthen}
\usepackage{comment}
\usepackage[colorinlistoftodos,textsize=tiny]{todonotes}
\usepackage{hyperref}

\renewcommand*{\thefootnote}{\fnsymbol{footnote}}
\sisetup{exponent-product = \cdot}
\newcommand{\tc}{\,\mbox{tc}\,}
\newcommand{\Epsilon}{\mathcal{E}}
\newcommand{\half}{\frac{1}{2}}
\newcommand{\overbar}[1]{\mkern 1.5mu\overline{\mkern-1.5mu#1\mkern-1.5mu}\mkern 1.5mu}
\let\oldfrac\frac
\renewcommand{\frac}[3][d]{\ifthenelse{\equal{#1}{d}}{\oldfrac{#2}{#3}}{\nicefrac{#2}{#3}}}

\title{VESPA\\Vaso per Esperimenti Su Plasmi ed Altro}
\author{Andrea Grossutti, mat. 1237344\\Alessandro Lovo, mat. 1236048\\Leonardo Zampieri, mat. 1237351}
\date{\today}

\begin{document}

\maketitle

\section{Aims}
Study the \emph{Vespa} experimental apparatus, and in particular:
\begin{itemize}[noitemsep]
    \item Model the vacuum system behavior, finding the characteristic parameters;
    \item Obtain the current-voltage and the current-temperature characteristics curves of the filament;
    \item Draw the voltage-current characteristics curves of the gas discharge, enhancing their behavior as varying pressure;
    \item Find the Paschen curve, both in DC and RF condition;
    \item Measurement of plasma parameters through a Langmuir probe, both in stationary conditions and via ionic-sonic wave propagation.
\end{itemize}


\section{Vacuum system}
The vacuum inside the VESPA vessel (a cylindrical  vessel, with a length of $\sim80\si{\centi\metre}$ and a radius of $\sim20\si{\centi\metre}$: $V\sim0.1\si{\metre^3}$) is obtained and kept thanks to a rotary pump and a turbomolecular pump. The vessel is not perfectly isolated and some small leaks affect the vacuum keeping. To study this phenomena, the vessel has been taken to a low pressure ($\sim 6 \cdot 10^{-5} \si{\milli\bar}$) and all the valves around have been closed. Isolating the chamber from the pumping system one can measure (thanks to a ionization pressure gauge) the pressure in the vessel as function of time. Effects as leaks and degasing contribute to an inflow in the chamber $F_0(p)$ that in principle could depend on the pressure. If it doesn't, $F_0$ can be estimated by performing a linear fit on the data ($P = a + b \cdot t$): $F_0 = V \cdot b$.

In all these measurements we consider a 5\% error on the pressure and a 0.5\si{\second} error on the time.

\begin{figure}[H]
  \centering
  \resizebox{0.5\textwidth}{!}{\import{img/vacuum/}{increase_all_fit.tex}} \hspace{-0.05\textwidth}
  \resizebox{0.5\textwidth}{!}{\import{img/vacuum/}{increase_all_residuals.tex}}
  \caption{Presure increasing in the chamber}
  \label{fig:increase_all}
\end{figure}

As can be seen from fig \ref{fig:increase_all} there is an evident trend in the residuals, proving that $F_0$ cannot be assumed constant throughout all the explored range of pressures. The simplest correction to this is considering a low pressure regime and a high pressure one.

\begin{figure}[H]
  \centering
  \resizebox{0.5\textwidth}{!}{\import{img/vacuum/}{increase_lowP_fit.tex}} \hspace{-0.05\textwidth}
  \resizebox{0.5\textwidth}{!}{\import{img/vacuum/}{increase_lowP_residuals.tex}}
  \caption{Low pressure regime}
  \label{fig:increase_lowP}
\end{figure}

\begin{figure}[H]
  \centering
  \resizebox{0.5\textwidth}{!}{\import{img/vacuum/}{increase_highP_fit.tex}} \hspace{-0.05\textwidth}
  \resizebox{0.5\textwidth}{!}{\import{img/vacuum/}{increase_highP_residuals.tex}}
  \caption{High pressure regime}
  \label{fig:increase_highP}
\end{figure}

Assuming a 5\% error on the volume,

\begin{gather*}
  F_0^\text{low} = (6.4 \pm 0.4) \cdot 10^{-8} \si{\pascal}\, \si{\metre^3}/\si{\second} \quad
  F_0^\text{high} = (4.5 \pm 0.4) \cdot 10^{-8} \si{\pascal}\, \si{\metre^3}/\si{\second}
\end{gather*}

Subsequently, the valve has been opened connecting the chamber to the pumping system. An exponential decay of the pressure is expected: $P(t) = (P_i - P_0)exp{-t/\tau} + P_0$, where $P_i$ is the starting pressure and $P_0$ the asymptotic pressure.

\begin{figure}[H]
  \centering
  \resizebox{0.5\textwidth}{!}{\import{img/vacuum/}{decrease_all_fit.tex}} \hspace{-0.05\textwidth}
  \resizebox{0.5\textwidth}{!}{\import{img/vacuum/}{decrease_all_residuals.tex}}
  \caption{Presure decreasing in the chamber}
  \label{fig:decrease_all}
\end{figure}

As can be seen from fig \ref{fig:decrease_all} there is an evident trend in the residuals, similarly as what seen before, and as before one should distinguish between the two regimes.

\begin{figure}[H]
  \centering
  \resizebox{0.5\textwidth}{!}{\import{img/vacuum/}{decrease_lowP_fit.tex}} \hspace{-0.05\textwidth}
  \resizebox{0.5\textwidth}{!}{\import{img/vacuum/}{decrease_lowP_residuals.tex}}
  \caption{Low pressure regime}
  \label{fig:decrease_lowP}
\end{figure}

\begin{figure}[H]
  \centering
  \resizebox{0.5\textwidth}{!}{\import{img/vacuum/}{decrease_highP_fit.tex}} \hspace{-0.05\textwidth}
  \resizebox{0.5\textwidth}{!}{\import{img/vacuum/}{decrease_highP_residuals.tex}}
  \caption{High pressure regime}
  \label{fig:decrease_highP}
\end{figure}

From the values of $\tau$ and $P_0$ one can estimate the effective pumping speed $S_e = V/\tau$, the inflow $F_0 = P_0 \cdot S_e$ and, given the nominal value of the pumping speed $S = 33 \si{\litre/\second}$, the conductance of the chamber-pump connection $C = 1/(1/S_e - 1/S)$.

\begin{table}[H]
  \centering
  \begin{tabular}{cccccc}
    \toprule
    regime & $S_e$ [\si{\litre/\second}] & $F_0$ [\si{\pascal}\, \si{\metre^3}/\si{\second}] & $C$ [\si{\litre/\second}] \\
    \midrule
    jhba & gfjha & bfgjlb & hasgjbha \\
    \bottomrule
  \end{tabular}
  \caption{Vacuum parameters}
  \label{tab:vacuum}
\end{table}


\section{Voltage-Current characteristic of the filament}
The filament inside the vessel is a tungsten filament with diameter $2r\sim0.25\si{{\milli\metre}}$ and length $L\sim10\si{\centi\metre}$. Combining Ohm law and emissivity rules, a theoretical characteristic curve can be obtained:
\[
    V = \frac{A^{\frac[f]{10}{7}}L}{\pi^{\frac[f]{13}{7}}r^{\frac[f]{23}{7}}(2\epsilon\alpha)^{\frac[f]{3}{7}}} \cdot I^{\frac[f]{13}{7}}
\]
where $\epsilon$ is the effective emissivity, $\alpha$ the StefanBoltzmann constant and $A$ a the resistivity proportional constant, such that the resistivity $\rho$ can be expressed as function of the temperature $T$ as
\[
    \rho(T) = AT^{\frac[f]{6}{5}}
\]

Pumping the vessel to a low density (...\todo{how much?}), the voltage-current characteristic curve of the filament has been measured, producing the following data:
\todo[inline]{Plot V-I filamento}

Fitting the data with a $V\propto I^{\frac[f]{13}{7}}$, the following parameters are found:
\begin{align}
    V &= m I ^{\frac[f]{13}{7}} + q \\
    m &= ...                        \\
    q &= ...
\end{align}
which lead to a value of
\[
    \epsilon = ...
\]
Finally, the estimated filament temperature as a function of the driven current can be found:
\[
    T = \frac{A^{\frac[f]{5}{14}}}{\pi^{\frac[f]{5}{7}}r^{\frac[f]{15}{14}}(2\epsilon\alpha)^{\frac[f]{5}{14}}} \cdot I ^{\frac[f]{5}{7}}
\]

\todo[inline]{Insert T-I plot}
\end{document}
