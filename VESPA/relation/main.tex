\documentclass[11pt,a4 paper]{article}
\usepackage{subfiles}

\usepackage{amsmath, amsthm}
\usepackage[english]{babel}
\usepackage[T1]{fontenc}
\usepackage[utf8]{inputenc}
\usepackage[margin=2cm]{geometry}
\usepackage{graphicx}
\usepackage{subfig}
\usepackage{caption}
\usepackage{siunitx}
\captionsetup{tableposition=top,font=small,width=0.8\textwidth}
\usepackage{booktabs}
\usepackage[table]{xcolor}
\usepackage[arrowdel]{physics}
\usepackage{mathtools}
\usepackage{tablefootnote}
\usepackage{amssymb}
\usepackage{enumitem}
\usepackage{multicol}
\setlist[description]{font={\scshape}} %style=unboxed,style=nextline
\usepackage{wrapfig}
\usepackage{float}
\usepackage{import}
\usepackage{floatflt}
\usepackage{url}
\usepackage{commath}
\usepackage{bm}
\usepackage[version=4]{mhchem}
\usepackage{nicefrac}
\usepackage{ifthen}
\usepackage{comment}
\usepackage[colorinlistoftodos,textsize=tiny]{todonotes}
\usepackage{hyperref}

\renewcommand*{\thefootnote}{\fnsymbol{footnote}}
\sisetup{exponent-product = \cdot}
\newcommand{\tc}{\,\mbox{tc}\,}
\newcommand{\Epsilon}{\mathcal{E}}
\newcommand{\half}{\frac{1}{2}}
\newcommand{\overbar}[1]{\mkern 1.5mu\overline{\mkern-1.5mu#1\mkern-1.5mu}\mkern 1.5mu}
\let\oldfrac\frac
\renewcommand{\frac}[3][d]{\ifthenelse{\equal{#1}{d}}{\oldfrac{#2}{#3}}{\nicefrac{#2}{#3}}}

\title{VESPA\\Vaso per Esperimenti Su Plasmi ed Altro}
\author{Andrea Grossutti, mat. 1237344\\Alessandro Lovo, mat. 1236048\\Leonardo Zampieri, mat. 1237351}
\date{\today}

\begin{document}

\maketitle

\section{Aims}
Study the \emph{Vespa} experimental apparatus, and in particular:
\begin{itemize}[noitemsep]
    \item Model the vacuum system behavior, finding the characteristic parameters;
    \item Obtain the current-voltage and the current-temperature characteristics curves of the filament;
    \item Draw the voltage-current characteristics curves of the gas discharge, enhancing their behavior as varying pressure;
    \item Find the Paschen curve, both in DC and RF condition;
    \item Measurement of plasma parameters through a Langmuir probe, both in stationary conditions and via ionic-sonic wave propagation.
\end{itemize}

\section{Vacuum system}
The vacuum inside the VESPA vessel is obtained and keep thanks to a rotary pump and a turbomolecular pump. The vessel is not perfectly isolated and some small leaks affect the vacuum keeping. To study this phenomena, the vessel has been taken to a low pressure (...\todo{insert pressure}) and all the valves around has been closed. Measuring (thanks to a ionization pressure gauge) the pressure in the vessel as function of the time, effect as leaks and degasing can be observer.

\todo[inline]{Highering pressure plot}

Subsequently, the valves has been opened; the turbomolecular pump acts to extract all the gas from the vessel, and therefore a pressure lowering can be observed.

\todo[inline]{Lowering pressure plot}

\section{}
\end{document}
