\documentclass[11pt,a4 paper]{article}
\usepackage{subfiles}

\usepackage{amsmath, amsthm}
\usepackage[english]{babel}
\usepackage[T1]{fontenc}
\usepackage[utf8]{inputenc}
\usepackage[margin=2cm]{geometry}
\usepackage{graphicx}
\usepackage{subfig}
\usepackage{caption}
\usepackage{siunitx}
\captionsetup{tableposition=top,font=small,width=0.8\textwidth}
\usepackage{booktabs}
\usepackage[table]{xcolor}
\usepackage[arrowdel]{physics}
\usepackage{mathtools}
\usepackage{tablefootnote}
\usepackage{amssymb}
\usepackage{enumitem}
\usepackage{multicol}
\setlist[description]{font={\scshape}} %style=unboxed,style=nextline
\usepackage{wrapfig}
\usepackage{float}
\usepackage{import}
\usepackage{floatflt}
\usepackage{url}
\usepackage{commath}
\usepackage{bm}
\usepackage[version=4]{mhchem}
\usepackage{nicefrac}
\usepackage{ifthen}
\usepackage{comment}
\usepackage[colorinlistoftodos,textsize=tiny]{todonotes}
\usepackage{hyperref}

\renewcommand*{\thefootnote}{\fnsymbol{footnote}}
\sisetup{exponent-product = \cdot}
\newcommand{\tc}{\,\mbox{tc}\,}
\newcommand{\Epsilon}{\mathcal{E}}
\newcommand{\half}{\frac{1}{2}}
\newcommand{\overbar}[1]{\mkern 1.5mu\overline{\mkern-1.5mu#1\mkern-1.5mu}\mkern 1.5mu}
\let\oldfrac\frac
\renewcommand{\frac}[3][d]{\ifthenelse{\equal{#1}{d}}{\oldfrac{#2}{#3}}{\nicefrac{#2}{#3}}}

\title{Positronium}
\author{Andrea Grossutti, mat. 1237344\\Alessandro Lovo, mat. 1236048\\Leonardo Zampieri, mat. 1237351}
\date{\today}

\begin{document}

\maketitle

\section{Aims}
\begin{itemize}
    \item Measure the ratio between the two and three photons decay of the Positronium;
    \item Measure the lifetime of the Positronium through the time distribution of the decays.
\end{itemize}



\section{Experimental setup}
The experimental setup consist in 4 inorganic scintillators; three coplanar (DET. 1,2,3) and a fourth (DET. 4) perpendicular to the plane.

The first three detectors are placed on a circumference in the center of which lies a \ce{^22Na} source, with an activity of around $380\si{kBq}$. These detectors are free to move around the circumference; in this session, two different configurations have been explored. To observe the two-photons decay, the detectors \#1 and \#2 have been aligned; instead, to observe the three-photons decay, the three coplanar detectors have been positioned at the vertices of an equilateral triangle.

Data are collected from the detectors through a electronic chain: a fan-in-fan-out quad module replies the signal of each detector and produces four copies of it; then, through a CFD, a trigger signal is produced.  The CFD trigger threshold has been set so that the background noise is discarded, while the interesting signals produce an output. As can be seen from fig. \ref{fig:oscilloscope}, the signals corresponding to the detection of the $511\si{\kilo\electronvolt}$ and $1275\si{\kilo\electronvolt}$ photons are clearly visible.

\begin{figure}[H]
    \centering
    \includegraphics[width=0.7\textwidth]{img/TEK0001.JPG}
    \caption{Trigger on the CFD (blue) of the signal from detector \#2 (yellow). It displays the two different types of peaks, respectively for the $511\si{\kilo\electronvolt} (\sim 50\si{\milli\volt})$ and $1275\si{\kilo\electronvolt} (\sim 125\si{\milli\volt})$ photons.}
    \label{fig:oscilloscope}
\end{figure}

Between the second and the third day, a technical problem required the substitution of the high voltage power supply; due to this, all thresholds have been re-set; moreover, in the middle of day 3 the high voltage power supply burned; it has been replaced and the thresholds had to be re-set once again. While in the first two days all the detectors saw the two peaks around an amplitude of respectively $100$ and $250\si{\milli\volt}$ (and therefore the thresholds had been set around $75\si{\milli\volt}$), in the third day detectors \#1 and \#3 saw the two peaks around an amplitude of respectively $50$ and $125\si{\milli\volt}$ (and therefore the thresholds have been set around $35\si{\milli\volt}$). For the detector \#4, finally, the threshold has been set around $200\si{\milli\volt}$, such that only the higher energetic photons produce a trigger.

The CFD is provided of two sets of microswitches, to adjust delay and width of the output signal. As verified thanks to the oscilloscope, the delay microswitches allow to adjust the time between the triggering of the system and the start of the logic signal, while the width microswitches can be used to adjust the time length of the logic signal.



\section{Apparatus calibration}
Due to the various problems, the apparatus calibration has been done many times; only the first calibration is here reported, since for the other we followed the same procedure.

The positions of the two peaks are known with very high precision:
\begin{table}[H]
    \centering
    \begin{tabular}{cccccccc}
        \toprule
        \ce{^22Na} Gamma radiations \\
        \midrule
        $511.0\,\si{\kilo\electronvolt}$ \\
        $1274.537\,\si{\kilo\electronvolt}$ \\
        \bottomrule\vspace{0.01cm}
    \end{tabular}
    \caption{Gamma radiation for \ce{^22Na} from NuDat, \url{https://www.nndc.bnl.gov/nudat2/decaysearchdirect.jsp?nuc=22NA\&unc=nds}}
    \label{tab:gammavalue}
\end{table}

Acquiring the energy spectra with the digitizer, the two peaks are clearly visible; by computing their centroids the horizontal axis can be linearly rescaled to be calibrated in energy. To fit the peaks, observing the variation of background before and after each peak, a gaussian plus a linear noise is used:
\begin{equation*}
    f(x) = \underbrace{a + bx}_\text{noise} + \epsilon\, e^{-\frac[f]{(x-\mu)^2}{2\sigma^2}}
\end{equation*}

The following result are found:
\begin{table}[H]
    \centering
    \begin{tabular}{cccccccc}
        \toprule
        Det. & $\mu$ & $\sigma$ & $\chi^2/dof$ \\
        \midrule
        \#1 & $4345.0 \pm 0.1$ & $142.7 \pm 0.8$ & $107/80$ \\
        & $10589 \pm 2$ & $233 \pm 2$ & $137/149$ \\
        \#2 & $4679.4 \pm 0.6$ & $143 \pm 1$ & $150/80$ \\
        & $11362 \pm 2$ & $249 \pm 2$ & $140/159$ \\
        \#3 & $3304.3\pm0.5$ & $112.3\pm 0.7$ & $66/61$ \\
        & $8021\pm 1$ & $205\pm3$ & $118/108$ \\
        \#4 & $3303.0\pm0.4$ & $112.4\pm0.5$ & $102/67$ \\
        & $8021\pm1$ & $205\pm3$ & $118/108$\\
        \bottomrule
    \end{tabular}
    \caption{Interpolation results}
    \label{tab:calibr1}
\end{table}

The similarity between the chi-squared value and the number of degree of freedom for all the fits, combined with the observation of the fit plots, guarantee the reliability of the interpolations. For each detector the rescaling factor are computed and thus the resolution $r$.

\begin{gather*}
    E = a \cdot \text{channel} + b \\
    r = \frac{\text{FWHM}_E}{\mu_E} = 2\sqrt{2\ln2} \,\frac{\sigma_E}{\mu_E} = 2\sqrt{2\ln2}\, \frac{a \cdot \sigma_c}{a \cdot \mu_c + b} = 2\sqrt{2\ln2} \,\frac{\sigma_c}{\mu_c + \frac[f]{b}{a}}
\end{gather*}

\begin{table}[H]
    \centering
    \begin{tabular}{cccccccc}
        \toprule
        Det. & $a [\si{\kilo\electronvolt}]$ & $b [\si{\kilo\electronvolt}]$ & $r_1$ & $r_2$ \\
        \midrule
        \#1 & $0.12228\pm0.00004$ & $-20.3\pm0.2$ & $8.03\pm0.05$ & $5.25\pm0.05$ \\
        \#2 & $0.11426\pm0.00004$ & $-23.7\pm0.2$ & $7.54\pm0.05$ & $5.25\pm0.05$ \\
        \#3 & $0.16188\pm0.00004$ & $-23.9\pm0.2$ & $8.38\pm0.05$ & $6.12\pm0.09$ \\
        \#4 & $0.16183\pm0.00004$ & $-23.5\pm0.2$ & $8.38\pm0.05$ & $6.12\pm0.09$ \\
        \bottomrule
    \end{tabular}
    \caption{Calibration result, $r_1$ and $r_2$ are the resolutions of the $511\si{\kilo\electronvolt}$ and $1275\si{\kilo\electronvolt}$ peaks respectively.}
    \label{tab:calib2}
\end{table}
The resolutions for the peaks are similar between the four detectors, as expected.

\begin{figure}[H]
    \centering
    \resizebox{0.8\textwidth}{!}{\import{img/}{det1_calibration.tex}}
    \caption{Calibration of detector 1: in blue the histogram of the detected signals, in red and green the peaks fits used for the calibration. Note the x axis properly calibrated in energy.}
    \label{fig:det1_calibr}
\end{figure}


Finally, even the TAC must be calibrated. This module returns a voltage signal whose amplitude is proportional to the time difference between the triggering of the \emph{start} and the \emph{stop} channels. To calibrate it, the two channels are connected to the same signal, inserting a delay module between the signal and the stop. Some measures are made with different delay, obtaining the following histograms (fig. \ref{fig:tachisto}).

\begin{figure}
    \centering
    \resizebox{0.8\textwidth}{!}{\import{img/}{tac_calibr.tex}}
    \caption{Measure collected from the TAC, with different delay, in about $1\si{\minute}$ of acquisition in day \#2 for the $2\gamma$ detection.}
    \label{fig:tachisto}
\end{figure}

Neglecting the zero-delay data (too close to the border of the histogram to efficiently find a centroid), centroids for the other peaks have been computed as the average of the registered values, with an associated error corresponding to the standard deviation of the data: a fit or a more precise statistic is meaningless due to the low number of channels in which data are spread on. Having no \emph{zero} point, i.e. even if the \emph{start} and the \emph{stop} signal are the same, cables and other electronic device can introduce a delay between them, only differences can be meaningfully defined. The calibration must be done separately for the $2\gamma$ and the $3\gamma$ detector, because the range of the TAC used in the two detections has been chosen different, to optimize the data collecting. A linear fit on the centroids returns the following results:

\begin{gather*}
    \Delta\text{channel}  = m \cdot \Delta t \\
    m_{2\gamma} = ( 117 \pm 1 )\si{\per\nano\second} \\
    m_{3\gamma} = ( 87.4 \pm 0.4 )\si{\per\nano\second}
\end{gather*}

\begin{figure}[H]
    \centering
    \resizebox{0.8\textwidth}{!}{\import{img/}{tac_calibr_fit.tex}}
    \caption{TAC calibration fit, for the $2\gamma$ setup.}
    \label{fig:tac_calibr_fit}
\end{figure}



\section{Digitizer dead time}
The digitizer has a dead time, i.e. when it is triggered, it takes some time to scan over all channel, compute integrals and register data. To estimate the dead time, during the three experimental days multiple tests have been done, measuring the frequency of different signals with both the digitizer (counting the number of registered events over the acquisition time) and a scaler module, which has a negligible dead time. The results found are summarized in table \ref{tab:rate}.

\begin{table}[H]
    \centering
    \begin{tabular}{llcccccc}
        \toprule
        Event && Real rate [\si{Hz}] (scaler)& Measured rate [\si{Hz}] (digit.) \\
        \midrule
        Det. \#1 &day 1& $9211 \pm 80$ & $6543 \pm 10$ \\
        Det. \#2 &day 1& $12248 \pm 80$ & $5997 \pm 10$\\
        Det. \#1\&\#2 &day 2& $3687 \pm 10$ & $3683 \pm 1$ \\
        Det. \#1\&\#2\&\#4 &day 2& $44 \pm 1$ & $43.9 \pm 0.2$ \\
        Det. \#1 &day 3& $12610 \pm 70$ & $7084 \pm 11$ \\
        Det. \#2 &day 3& $12460 \pm 40$ & $6001 \pm 10$ \\
        %Coincidences det. \#1-\#2 & $4740 \pm 20$ & $$ \\
        Det. \#1\&\#2\&\#4 &day 3& $60 \pm 3$ & $59.8 \pm 0.1$ \\
        \bottomrule
    \end{tabular}
    \caption{Event rate}
    \label{tab:rate}
\end{table}

The PDF of the time between two consecutive decays follows an exponential distribution:
\begin{equation*}
    p(t) = \nu e^{-\nu \,t} \qquad \rightarrow \qquad <t> = \int_0^{\infty} t\,p(t) \dd{t} = \frac{1}{\nu}
\end{equation*}
where $\nu$ is the event frequency. If the system has a dead time $\tau$, two events are both picked only if they differ of a time larger than $\tau$; the measured frequency is therefore:
\begin{equation*}
    \frac{1}{\nu^\star} = \frac{\int_{\tau}^{\infty} t \,p(t) \dd{t}}{\int_{\tau}^{\infty} p(t) \dd{t}} =\frac{\nu\,\tau + 1}{\nu}
\end{equation*}

Plotting the data in table \ref{tab:rate}, we obtain the plots in figs. \ref{fig:deadtime:fit}, \ref{fig:deadtime}

% \begin{figure}
%     \centering
%     \subfloat[][Correlation between real and measured rate, and trend.]
%     {\resizebox{0.7\textwidth}{!}{\import{img/}{dead_time.tex}} \label{fig:deadtime:fit}} \\
%     \subfloat[][Ratio of lost events.]
%     {\resizebox{0.7\textwidth}{!}{\import{img/}{dead_time.ratio.tex}}}
%     \caption{Dead time effects.}
%     \label{fig:deadtime}
% \end{figure}

\begin{figure}[H]
  \centering
  \resizebox{0.7\textwidth}{!}{\import{img/}{dead_time.tex}}
  \caption{Correlation between real and measured rate, and hyperbolic trend.}
  \label{fig:deadtime:fit}
\end{figure}

\begin{figure}[H]
  \centering
  \resizebox{0.7\textwidth}{!}{\import{img/}{dead_time.ratio.tex}}
  \caption{Ratio of lost events.}
  \label{fig:deadtime}
\end{figure}

As can be seen from the graph in fig \ref{fig:deadtime}, the dead time isn't constant, but is the result of a stochastic process and can vary at different rates. However, a coarse trend can be estimated (fit in fig. \ref{fig:deadtime:fit}) computing the order of magnitude of dead time:
\begin{equation*}
    \tau \sim 6\cdot10^{-5} \si{\second}
\end{equation*}



\section{Two photons events}

\begin{figure}[H]
    \centering
    \subfloat[][Placement of the detectors]
    % {\def\svgwidth{0.4\textwidth}\import{img/}{rivel180.pdf_tex} \label{fig:rivel180:placement}} \quad
    {\resizebox{0.4\textwidth}{!}{\import{img/}{rivel180.pdf_tex}} \label{fig:rivel180:placement}} \quad
    \subfloat[][Electronic scheme]
    % {\def\svgwidth{0.4\textwidth}\import{img/}{electro180.pdf_tex}}
    {\resizebox{0.4\textwidth}{!}{\import{img/}{electro180.pdf_tex}}}
    \caption{Two photons setup.}
    \label{fig:rivel180}
\end{figure}

To detect the two-photons events, the detectors have been placed as in fig. \ref{fig:rivel180:placement}. The electronics have been set to be triggered with the coincidente between det. \#1, \#2 and \#4. However, we expect the detector \#4 to observe the photons shortly before the other two detectors, because of the lifetime of the positronium. To prevent \emph{false negative}, the width of the CFD of the detector \#4 has been set to a high value to produce a long logic signal.

Moreover, the system has been set to measure via the TAC the time difference between the signal from det. \#4 and the one from det. \#1\&\#2.

After the acquisition, a data filtering has been implemented; the events whose sum of the energies measured by detector \#1 and \#2 differs from $1022\si{\kilo\electronvolt}$ for more than 10\% (i.e. about two time the resolution) are discared. The spectra with and without the filter are showed in fig. \ref{fig:det1:2gamma} and the followings.

\begin{figure}[H]
    \centering
    \resizebox{0.64\textwidth}{!}{\import{img/}{det1.2gamma.tex}}
    \caption{Detector 1. Note the absence of the $1275\si{\kilo\electronvolt}$ peak.}
    \label{fig:det1:2gamma}
\end{figure}

\begin{figure}[H]
    \centering
    \resizebox{0.64\textwidth}{!}{\import{img/}{det2.2gamma.tex}}
    \caption{Detector 2. Note the absence of the $1275\si{\kilo\electronvolt}$ peak.}
    \label{fig:det2:2gamma}
\end{figure}

\begin{figure}[H]
    \centering
    \resizebox{0.64\textwidth}{!}{\import{img/}{det4.2gamma.tex}}
    \caption{Detector 4. Note the absence of the $511\si{\kilo\electronvolt}$ peak.}
    \label{fig:det4:2gamma}
\end{figure}

Acquiring the time with the TAC, what we obtain is the convolution of the decay law with the statistical gaussian error. At the end, what we see is a peaks centered in an arbitrary value (determined by the electronic configuration of the apparatus, the cables used and the delay introduced) which on its left tail shows a gaussian trend, while on the right the gaussian contribute is negligible and the exponential trend is visible. Fitting the right tail with an exponential it is possible to find the parameters of the decay law, in particular the mean lifetime.

\begin{figure}[H]
    \centering
    \resizebox{0.8\textwidth}{!}{\import{img/}{tac.2gamma.tex}}
    \caption{TAC for the 2$\gamma$ decay, with the exponential fit.}
    \label{fig:tac:2gamma}
\end{figure}

Fitting the right tail of the peak, an estimation of the mean life can be found:

\begin{equation*}
    \tau_{2\gamma} = ( 0.91 \pm 0.01) \si{\nano\second}
\end{equation*}

\section{Three photons events}

\begin{figure}[H]
    \centering
    \subfloat[][Placement of the detectors]
    % {\def\svgwidth{0.4\textwidth}\import{img/}{rivel120.pdf_tex} \label{fig:rivel120:placement}} \quad
    {\resizebox{0.4\textwidth}{!}{\import{img/}{rivel120.pdf_tex}} \label{fig:rivel120:placement}} \quad
    \subfloat[][Electronic scheme]
    %{\def\svgwidth{0.4\textwidth}\import{img/}{electro120.pdf_tex}}
    {\resizebox{0.4\textwidth}{!}{\import{img/}{electro120.pdf_tex}}}
    \caption{Three photons setup.}
    \label{fig:rivel120}
\end{figure}
To detect the three-photons events, the detectors have been placed as in fig. \ref{fig:rivel120:placement}. The optimal configurations would provide a triggering on the coincidences between all the four detectors. However, the three-photons events are very rare, and triggering on all the detectors would produce a too-low number of data to make a meaningful statistic, even with a full-night acquisition. Therefore the trigger has been set on the coincidences between the det. \#1, \#2 and \#3, making an a-posteriori coincidence with the detector 4 only for the timing studies.
\begin{figure}[H]
    \centering
    \resizebox{0.8\textwidth}{!}{\import{img/}{sum.3gamma.tex}}
    \caption{Sum of the energies observed from each detector}
    \label{fig:sum:3gamma}
\end{figure}
Observing fig. \ref{fig:sum:3gamma}, in which the sum of the energies detected from the three coplanar detectors is plotted, we clearly see the peak we're interested in, at $1022\si{\kilo\electronvolt}$. All around this peak there is a lot of noise, given mainly by the way-more-probable 2-photons events. To select only the events we're interested in, a filter is applied: each event that violate one or more of the following conditions is discarded.
\begin{itemize}[noitemsep]
    \item The sum of the energies from the three detectors must be $1022\si{\kilo\electronvolt}$, with a $10\%$ tolerance (about $3\sigma$);
    \item Each detector must detect an energy around $340\si{\kilo\electronvolt}$, with a $10\%$ tolerance.
\end{itemize}

As can be seen from fig. \ref{fig:alldet:3gamma}, all the detectors show a peak around $340\si{\kilo\electronvolt}$, as expected; their spectra are similar so the symmetry is respected.

Finally, the TAC spectrum is computed. During the acquisition no constraint on coincidences with detector \#4 has been imposed, so most of the recorded events lack of TAC start triggering. Excluding the first and the last bin, containing the not-started and the not-stopped events, a filtered TAC spectrum can be computed (fig \ref{fig:tac:3gamma}).

\begin{figure}[H]
    \centering
    \resizebox{0.8\textwidth}{!}{\import{img/}{alldet.3gamma.tex}}
    \caption{Filtered energy spectra of the detectors. The different bin heights is due to different binning width, itself due to different calibration parameters.}
    \label{fig:alldet:3gamma}
\end{figure}

\begin{figure}[H]
    \centering
    \resizebox{0.8\textwidth}{!}{\import{img/}{tac.3gamma.tex}}
    \caption{TAC for the three photons decay.}
    \label{fig:tac:3gamma}
\end{figure}


This spectrum has a too low statistic to allow any meaningful analysis, even if the acquisition runned for about 23 hours. The three-photons decay is a very rare event and only with a very long acquisition time plus a powerful filter to discard two-photons noise some results can be obtained.

\section{Rate}

Applying the filter to the two photons and the three photons spectra (using the spectra recorded without the det. \#4 coincidence), the obtained frequency are:
\begin{equation*}
    f_{2\gamma} = (2709 \pm 1)\si{\hertz} \qquad f_{3\gamma} = (0.259 \pm 0.002)\si{\hertz}
\end{equation*}
where the errors are computed assuming a Poissonian distribution. The two acquisition have, before the filtering, a frequency of respectively $(3674\pm1)\si{\hertz}$ and $(2.325\pm0.006)\si{\hertz}$: this means that for both the acquisition the dead time effect on the computed frequencies can be neglected. To estimate the ratio between the two decays, even the geometry of the system must be considered.

A first rough estimate can be done with a simple model: given $r$ the detector radius ($r\sim5\si{\centi\meter}$) and $d$ the distance of the detector from the source ($d\sim18\si{\centi\meter}$), we can roughly estimate that:
\begin{itemize}[noitemsep]
    \item For the two photons decay:
    \begin{itemize}[noitemsep]
        \item The first photons is detected only if it enters in one of the two detectors, so with a probability of $\frac[f]{2\pi r^2}{4\pi d^2}$;
        \item For momentum conservation law, if the first enters a detector, the second must enter the other one;
    \end{itemize}
    \item For the three photons decay:
    \begin{itemize}[noitemsep]
        \item As in the two photons case, the first photon is detected with a probability of $\frac[f]{3 \pi r^2}{4\pi d^2}$;
        \item This time even the second photon is free, so the detection probability is $\frac[f]{2 \pi r^2}{4\pi d^2}$;
        \item The third photon is constrained in the plane of the detectors (in a third of the plane actuallly): its probability it's something like  $\frac[f]{2r}{2\pi d/3}$;
    \end{itemize}
\end{itemize}
The correction factor is therefore about:
    \begin{equation}
        c_f \sim \frac{\frac{2 \pi r^2}{4\pi d^2}}{\frac{3 \pi r^2}{4\pi d^2}\cdot\frac{2 \pi r^2}{4\pi d^2}\cdot \frac{6r}{2\pi d}} = \frac{4\pi d^3}{9r^3} \sim 65
    \end{equation}
which lead to a ratio of about
\begin{equation}
    R \sim \frac{f_{2\gamma}}{f_{3\gamma}} \,\frac{1}{c_f} \sim 160
\end{equation}

\subfile{sim}

\section{Conclusions}
With respect to the intended aims,
\begin{itemize}[noitemsep]
    \item The shape of the spectra is in agreement with the theory; the expected energy peaks are clearly present where they're supposed to be;
    \item The calibration of the detectors has been made with success; the peaks resolutions are compatible between the four detector;
    \item The dead-time trend has been estimated, giving a clue in understanding if a measured frequency is meaningful or not;
    \item An interesting value for the ratio between the two and three photons decay is obtained; Moreover, the result $R=(150\pm15)$ matches (in order of magnitude) the value predicted by the theory ($R_{th}=373$); we didn't expect a better precision in the estimation of $R$, due to the uncertainty on the correction factor and the neglection of detector efficiency;
    \item Concerning the mean lifetime of the Positronium:
    \begin{itemize}[noitemsep]
        \item For the orthopositronium, the rarity of the event and the noise from parapositronium hindered the collection of a sufficient statistic;
        \item For the parapositronium, we obtained $\tau = ( 0.91 \pm 0.01) \si{\nano\second}$. This result isn't compatible with theory, from which we expect a lower lifetime. There are two possible explanations for this result:
        \begin{itemize}[noitemsep]
            \item Material effect: the mean lifetime of substances can be affected by the material in which they decay. In some studies concerning the decay of parapositronium in materials, values compatible with our result are obtained\footnote{See for example the second component in \emph{McGervey, John D. and Walters, Virginia F.: Correlation between Lifetime and Momentum for Positron Annihilations in Teflon. Phys. Rev. B 2(1970), \url{https://link.aps.org/doi/10.1103/PhysRevB.2.2421}}};
            \item Electronic effect: delay and errors in the electronics chain can be responsible for the exponential decay, hiding smaller parapositronium-decay exponential shape.
        \end{itemize}
    \end{itemize}
\end{itemize}

The experience can be however considered a success: despite the multiple problems with the HV module, we managed to complete a meaningful data acquisition and some interesting results.
\end{document}
