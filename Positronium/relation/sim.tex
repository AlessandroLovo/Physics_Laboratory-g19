\documentclass[main.tex]{subfiles}


\begin{document}

\subsection{Simulations}

In order to compute how many of the three-photons events we detect out of all the three-photons decays we performed a Montecarlo simulation. If we denote as $E$ the total energy of the three photons, we randomly divided the segment $[0,E]$ into three parts $E_1,E_2,E_3$. Since for a photon the modulus of its momentum is its energy and we assume the decay happens at rest, i.e. the momenta should sum to 0, $E_1,E_2,E_3$ must obey the trangular inequality:

\begin{equation}
  \sum_i E_i \geq 2 \max_i E_i
\end{equation}

% \begin{gather*}
%   \theta_{12} = \arccos\frac{E_3^2 - E_1^2 - E_2^2}{2E_1E_2} \\
%   \theta_{13} = \pi - \frac{\theta_{12}}{2} - \arccos(\left(\frac{E}{E_3} - 1\right)\cos\frac{\theta_{12}}{2}) \\
%   \theta_{23} = 2\pi - \theta_{12} - \theta_{13}
% \end{gather*}

So, by discarding invalid triplets, we obtain a uniform distribution on all the possible energy triplets. The resulting distribution of the angles between the photons is non trivial (fig \ref{fig:sim:ang_corr})
Now we can simulate also the three detectors and trigger on the triple coincidences, obtaing for each detector an energy spectrum like in fig \ref{fig:sim:det}.\\
Among $10^7$ three-photons decays, 4927 events triggered all the detectors, so we can give an estimate of the correction coefficient to the observed rate:

\begin{equation*}
  c_{3\gamma} := \frac{\text{\# detected events}}{\text{\# events}} \approx 5*10^{-4}
\end{equation*}

With a much simpler calculation one can obtain the correction coefficient for the two-photons decay:
\begin{equation*}
  c_{2\gamma} \approx 3.7*10^{-2}
\end{equation*}

We can now take the ratio between the two in order to find the right coefficient to match three-photons decay and two-photons ones:
\begin{equation*}
  c := \frac{c_{3\gamma}}{c_{2\gamma}} = \frac{\text{\# detected 3$\gamma$ events}}{\text{\# detected 2$\gamma$ events}} \frac{\text{\# 2$\gamma$ events}}{\text{\# 3$\gamma$ events}} \approx 1.4*10^{-2}
\end{equation*}

This coefficient depends on the distance detector-source that we measured to be roughly \si{18}{cm}. From the simulation we see that $c$ varies less than 10\% for distances between \si{17.5}{cm} and \si{18.5}{cm}

\begin{figure}[H]
    \centering
    % \subfloat[][]
    % {\def\svgwidth{0.3\textwidth}\import{img/}{simulated_energy_correlation.tex} \label{fig:sim:e_corr}} \quad
    \subfloat[][]
    {\resizebox{0.4\textwidth}{!}{\import{img/}{simulated_angular_correlation.tex}} \label{fig:sim:ang_corr}} \quad
    \subfloat[][]
    {\resizebox{0.4\textwidth}{!}{\import{img/}{simulated_detector.tex}} \label{fig:sim:det}}
    \caption{}
    \label{fig:sim}
\end{figure}

\end{document}
