\documentclass[main.tex]{subfiles}


\begin{document}

\subsection{Simulations}

The correction factor is better estimated through a Montecarlo simulation. If we denote as $E$ the total energy of the three photons, we randomly divided the segment $[0,E]$ into three parts $E_1,E_2,E_3$.
\todo{Should we say that our simulation's results are coherent with published paper?}
%Since for a photon the modulus of its momentum is its energy and we assume the decay happens at rest, i.e. the momenta should sum to 0, $E_1,E_2,E_3$ must obey the trangular inequality:
%
%
%\begin{equation}
%  \sum_i E_i \geq 2 \max_i E_i
%\end{equation}
%
% \begin{gather*}
%   \theta_{12} = \arccos\frac{E_3^2 - E_1^2 - E_2^2}{2E_1E_2} \\
%   \theta_{13} = \pi - \frac{\theta_{12}}{2} - \arccos(\left(\frac{E}{E_3} - 1\right)\cos\frac{\theta_{12}}{2}) \\
%   \theta_{23} = 2\pi - \theta_{12} - \theta_{13}
% \end{gather*}

Discarding triplets that doesn't allow momentum conservation, we obtain a uniform distribution on all the possible energy triplets. %The resulting distribution of the angles between the photons is non trivial (fig \ref{fig:sim:ang_corr})
Simulating also the three detectors and triggering on the triple coincidences, the ratio of observed events results:

\begin{equation*}
  c_{3\gamma} := \frac{\text{detected events}}{\text{total events}} \sim 5\cdot10^{-4}
\end{equation*}

\begin{figure}[H]
  \centering
  \resizebox{0.8\textwidth}{!}{\import{img/}{simulated_angular_correlation.tex}}
  \caption{Correlation of the angles between first and second (x axis) and first and third (y axis) simulated photon.}
  \label{fig:sim:ang:corr}
\end{figure}

With a similar calculation the ratio of observed events for the two-photons decay can be obtained:
\begin{equation*}
  c_{2\gamma} \sim 3.7\cdot10^{-2}
\end{equation*}

Therefore the correction factor results:
\begin{equation}
  c_f = \frac{c_{2\gamma}}{c_{3\gamma}} = 71\pm7
\end{equation}
where the errors is estimated thanks to the simulations, remembering the uncertainty on $R$ and $r$.
This lead to a ratio of about:
\begin{equation}
  R = 150 \pm 15
\end{equation}


%\begin{figure}[H]
%    \centering
    % \subfloat[][]
    % {\def\svgwidth{0.3\textwidth}\import{img/}{simulated_energy_correlation.tex} \label{fig:sim:e_corr}} \quad
%    \subfloat[][]
%    {\resizebox{0.4\textwidth}{!}{\import{img/}{simulated_angular_correlation.tex}} \label{fig:sim:ang_corr}} \quad
%    \subfloat[][]
%    {\resizebox{0.4\textwidth}{!}{\import{img/}{simulated_detector.tex}} \label{fig:sim:det}}
%    \caption{}
%    \label{fig:sim}
%\end{figure}

\end{document}
