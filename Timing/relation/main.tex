\documentclass[11pt,a4 paper]{article}
\usepackage{subfiles}

\usepackage{amsmath, amsthm}
\usepackage[english]{babel}
\usepackage[T1]{fontenc}
\usepackage[utf8]{inputenc}
\usepackage[margin=2cm]{geometry}
\usepackage{graphicx}
\usepackage{subfig}
\usepackage{caption}
\usepackage{siunitx}
\captionsetup{tableposition=top,font=small,width=0.8\textwidth}
\usepackage{booktabs}
\usepackage[table]{xcolor}
\usepackage[arrowdel]{physics}
\usepackage{mathtools}
\usepackage{tablefootnote}
\usepackage{amssymb}
\usepackage{enumitem}
\usepackage{multicol}
\setlist[description]{font={\scshape}} %style=unboxed,style=nextline
\usepackage{wrapfig}
\usepackage{float}
\usepackage{import}
\usepackage{floatflt}
\usepackage{url}
\usepackage{commath}
\usepackage{bm}
\usepackage[version=4]{mhchem}
\usepackage{nicefrac}
\usepackage{ifthen}
\usepackage{comment}
\usepackage[colorinlistoftodos,textsize=tiny]{todonotes}
\usepackage{hyperref}

\renewcommand*{\thefootnote}{\fnsymbol{footnote}}
\sisetup{exponent-product = \cdot}
\newcommand{\tc}{\,\mbox{tc}\,}
\newcommand{\Epsilon}{\mathcal{E}}
\newcommand{\half}{\frac{1}{2}}
\newcommand{\overbar}[1]{\mkern 1.5mu\overline{\mkern-1.5mu#1\mkern-1.5mu}\mkern 1.5mu}
\let\oldfrac\frac
\renewcommand{\frac}[3][d]{\ifthenelse{\equal{#1}{d}}{\oldfrac{#2}{#3}}{\nicefrac{#2}{#3}}}

\title{Timing}
\author{Andrea Grossutti, mat. 1237344\\Alessandro Lovo, mat. 1236048\\Leonardo Zampieri, mat. 1237351}
\date{\today}

\begin{document}

\maketitle

\section{Aims}
\begin{itemize}
    \item Energy calibration of the organic scintillators and calculation of the energy resolution from the analysis of the Compton edge;
    \item Optimization of the external delay of the analogue CFTD to obtain the best time resolution;
    \item Study of the time resolution behaviour as a function of the energy;
    \item Comparison between the timing resolutions obtained from analogue and digital treatment of the signals;
    \item Measurement of the speed of light
\end{itemize}

\section{Experimental setup}
The experimental setup consist of two collinear organic scintillators, mounted on a sledge, and a \ce{^22Na} source collimated between two lead bricks.

Data are collected from the detectors through a electronic chain: a fan-in-fan-out quad module replies the signal of each detector and produces four copies of it; then, through a CFD, a trigger signal is produced. The CFD trigger threshold has been set so that the background noise is discarded, while the interesting signals produce an output.

\section{Apparatus calibration}
Both the detectors and the TAC need to be calibrated. Firstly, let's calibrate the detector.

A spectra for each detector is acquired; due to the composition of the detector, the photopeaks are negligible and only the compton effect are detected. Through the position of the compton edge, the calibration can be done. 
\end{document}
